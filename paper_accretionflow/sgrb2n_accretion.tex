\documentclass[twocolumn]{aastex61}
\usepackage{datetime} % TODO: remove this after submission
\input{preface}
\newcommand{\dsgrb}{\ensuremath{8.4\textrm{~kpc}}\xspace}
\newcommand{\percent}{\%\xspace}

\begin{document}
\title{Title: An extraordinary accretion flow in Sgr B2 North}
\titlerunning{Sgr B2 ALMA LB}
\authorrunning{Ginsburg et al}
\input{authors}
\begin{abstract}
    This is an abstract
\end{abstract}



\section{Introduction}

Sgr B2 N

High-mass protoclusters

High-mass stars


\section{Observations}


Figure \ref{fig:rgb} shows the inner portions of Sgr B2 N and M in three-color
composite, including the 7 mm, 3 mm, and 1 mm data.  Dusty and ionized regions
take on distinctive colors.

\FigureTwo
{figures/SgrB2M_RGB.pdf}
{figures/SgrB2N_RGB.pdf}
{Color composite images showing Sgr B2 M (left) and Sgr B2 N (right) in 7 mm (red),
3 mm (green), and 1 mm (blue) continuum emission.  Both images are normalized to 99.99\%
of the peak emission, highlighting the much weaker peak emission at 1 mm in Sgr B2 N.}
{fig:rgb}{1}{3.5in}

\section{Results}

There are several notable features that make this accretion flow unique within our Galaxy:

\begin{itemize}
    \item Four spiral-shaped accretion filaments point toward a common central hub
    \item The accretion filaments are \emph{optically thick} in the 1 mm continuum
        along their centers
    \item There is a central hypercompact \hii region that cannot be seen at 1 mm
        or shorter wavelengths, yet some of its light leaks out in H41$\alpha$ emission.
\end{itemize}


title?: The radiation environment in the most massive forming cluster in the local group

\section{Results: Description of the Sgr B2 N region}
\subsection{The Veil}
\label{sec:theveil}
Toward Sgr B2 N, we observe a transition from likely optically thin to
certainly optically thick emission between 3 and 1 mm.  The overall structure
of the region, as seen at 3 mm, is a central main north-south filament, another
long and bright filament toward the east, and several smaller filaments at
various angles along the main ridge.  At 1 mm, there is a narrow region along
the main ridge with $T_{B,1 mm} < T_{B,3 mm}$, which is a clear indication
that the material is optically thick and centrally heated.

In order to reach an optical depth $\tau\geq1$ at 1-3 mm, a huge quantity of
dust is required.  Adopting an extrapolated dust opacity $\kappa_\nu = 0.0018
(\nu/96 \mathrm{GHz})^{\beta}$ cm$^2$ g$^{-1}$, the required column density is
$N(\hh) = 1.2\ee{26} (\nu/96 \mathrm{GHz})^{-\beta}$ \persc, or $N(\hh) \sim
2\ee{25} - 1\ee{26}$ \persc.  In a circular beam with $r=0.05$\arcsec (425 AU),
this translates to $\sim6-35$ \msun per beam.  The opacity is highly uncertain,
as it is likely that, given the high observed temperatures, most of the grain
mantles have been evaporated, so it is safest to regard these column density
and mass estimates as lower limits.

The detection of an optically thick surface in the range 1-3 mm is consistent
with the predictions from the SED modeling of \citet{Schmiedeke2016a}, who
suggested that the dust should become optically thick on 0.5\arcsec scales
around 1 mm.

The central source, Sgr B2 N K2, is behind this veil.  At 3 mm, it has a peak
brightness temperature of $\gtrsim1000$ K toward the central source, but at this
location, the brightest position measures only $\sim200$ K.  (CHECK) The 1 mm 
source centroid is also offset from the 3 mm centroid, indicating that we are
only seeing a component or reflection of the illuminating source at 1 mm.
K2 is behind the veil.

K2 is also at the intersection between the main north-south filament and the
eastern filament.  It is reasonable to suppose that it is the central
accretion point for the whole $>10^4$ \msun system.

We are able to see some component of the K2 source at 1 mm, which indicates
either that the source is \emph{much} more luminous at that position (unlikely)
or that there is a small gap in the foreground cloud at that position.
In either case, though, we can see an absorption spectrum through the foreground
veil toward this line-of-sight.



% \paragraph{Ruling out synchrotron emission} The spectral index from 3 mm to 1
% mm is negative, with $\alpha\approx-1.6$.  This is steeper than any synchrotron
% source (right?).


\subsection{The nature of K2}
\label{sec:natureofk2}
The source K2 at the center of Sgr B2 N is easily detected at wavelengths 3 mm
and longer, but nearly disappears at 1 mm and shorter wavelengths behind the
veil described in Section \ref{sec:theveil}.

The SED of K2 is unique because of its location behind the veil.  Unlike other
millimeter cores or \hii regions, the source flux declines below 1 mm.  We are
left with long-wavelength data only to interpret this source.

% TODO: measure these
The 3 mm flux density is $\sim25$ mJy, while the 2012 7 mm flux density was
$\sim20$ mJy.  Assuming no variability, these numbers are consistent with a
mostly optically thin \hii region ($\alpha\approx0.25$).  The brightness
temperatures at these wavelengths (3300 K and 1300 K at 7 mm and 3 mm
respectively) tell the same story; these temperatures are much higher than
expected for dusty sources and the rise toward longer wavelengths suggests a
$\sim5000-10^4$ K medium that is becoming optically thicker.
K2 is perhaps best described as a dust-obscured hypercompact \hii region.

K2 is surrounded by emission at both 3 and 7 mm.  There is emission to the
northwest at both wavelengths, while at 3 mm there is emission in all
directions.  The high brightness temperature of the northwest emission ($T_{B,7
mm}\gtrsim500$ K) hints that it is again free-free emission, while the
nondetection of much of the extended 3 mm emission suggests that it is
predominantly dust.  Unfortunately, the presence of the free-free emission complicates
the SED measurement of K2 at longer wavelengths, as the 1.3 cm beam from De Pree+
includes contributions from this extended component.

\paragraph{The SiO Maser}
There is a single SiO J=2-1 v=2 maser detected in Sgr B2 N, as previously
reported by \citet{Higuchi2015a}.  It is close to K2,
but not associated with the compact point source; they are separated
by 0.192\arcsec (1600 AU).  It is along the
line of sight to the nearby \hii region described above.
The maser location is shown with a white X in Figure \ref{fig:sioflow}; it is
not directly at the center of the bipolar outflow, so it may not indicate
the location of the driving source of the outflow.

As noted by \citet{Higuchi2015a}, the v=2 SiO maser is rarely detected.  This
confirmed detection, with peak intensity 0.44 Jy/beam, is unique.  We also
confirm that there is no sign of maser emission from any of the other
vibrational states of SiO, either in the J=2-1 or the J=5-4 transitions
or from any isotopologues of SiO in the observe bands.

\paragraph{Vibrationally Excited HCN}
We detect emission from the HCN v3=1 J=1-0 state, which has an upper state
energy level $E_U = 3021$ K.  This emission is confined to the inner 0.4\arcsec
around K2-\hii; it is centered on the extended emission, \emph{not} around the
bright 3 mm compact source.  The SiO J=2-1 v=2 maser is very close to the center
of the vibrationally excited HCN emission.  The nearby HCN v1=1 line, with
$E_U=4769$ K, is also detected, though more weakly.
{\color{red} Investigate these lines.  For higher-J HCN, there are apparently
selection rules that can be used to distinguish between collisional and radiative
excitation.  This might be a case where we can definitively measure the local
radiation field.}

\subsection{Star Formation and Fragmentation in the Veil}
As noted in Section \ref{sec:theveil}, the inverted 1 mm - 3 mm SED in many
spatial pixels across the `veil' implies the presence of high optical depths
and correspondingly extremely high column densities.
Despite these extraordinary column densities, we do not observe any clear
signs of fragmentation outside of the innermost $\sim1000$ AU; there are no
point sources in the inner 3 \arcsec region and the gas emission appears
continuous across the field of view.

Nominally, a $>6$ \msun blob of gas residing in a spherical region $r<450$ AU
across ($n>10^9$ \percc) is highly Jeans unstable even at $T\gtrsim 500$ K,
with $M_J \sim 0.5$ \msun at such high densities.  However, we can assume
instead that the material is spread out along a line-of-sight distance
comparable to the observed length scale of the filaments, $\sim1-3\arcsec$ or
8000-25000 AU, which would lower the density to $n\sim10^{7.5-8}$ \percc -
still an extraordinarily high density for such large scales.   At these
densities, the Jeans mass rises to $M_J\sim2-3$ \msun.


\subsection{Inflow and Outflow}
The streams pointing in to K2 appear to be inflow channels.  However, because
of the optical depth in the continuum noted in Section \ref{sec:theveil}, it is
not possible to characterize the kinematics of these channels at 1 mm.



\subsection{The gas}
Sgr B2 N is also know as the ``Molecular Heimat'', the region in our Galaxy
host to the greatest number of detected molecules.  We detect this region as
a $\sim2$\arcsec radius circle centered on K2.  It contains a forest of lines
with highly complicated chemistry and kinematics.

In this section, we break down the observed material into spatial and kinematic
components that we are able to visually distinguish.

\paragraph{The NW/SE outflow}
There is a giant outflow detected in SiO J=5-4 and several other lines (e.g., SO $6_5-5_4$).
It has several surprising features illustrated in Figure \ref{fig:sioflow}:
\begin{enumerate}
    \item The outflow lobes are centered on a point at least 0.2\arcsec (1600 AU)
        from K2, suggesting that K2 is not the dominant accretion source.
    \item The lobes correspond to regions conspicuously devoid of continuum
        emission, which implies that this outflow is a dominant force shaping
        the gas.
\end{enumerate}
The SiO has broader width than other lines, confirming that it is a kinematically
distinct feature and not simply part of the overall accretion flow.

\Figure{figures/sgrb2n_sio_outflow.pdf}
{An image of the SiO J=5-4 outflow from Sgr B2 N.
The contours show the redshifted (moment 0 from 77 to 100 \kms) and blueshifted
(moment 0 from 20 to 50 \kms) lobes at levels of 100, 200, 300, 400, and 500
mJy \perbeam in $0.083\arcsec\times0.042\arcsec$ beams at PA=80.5$\deg$.  The
greyscale shows the 1 mm continuum, robust -2 weighted image.  The yellow
contours show the 3 mm continuum at levels of 5, 10, and 20 mJy \perbeam in
$0.057\arcsec\times0.042\arcsec$ beams at PA=37.9$\deg$.
The giant outflow apparently does not point back to the central K2 source, hinting
that the dominant accretion source remains undetected.
The outflow lobes correspond to regions of substantially lower continuum emission,
suggesting that the outflows have a dramatic effect on the structure of the region
and are the dominant force shaping it.
}
{fig:sioflow}{1}{7in}

By contrast, there is no clear bipolar outflow in the NE/SW direction.
There is a large-scale line feature spanning a narrow 
velocity range in this direction [ it is difficult to determine because it shows up
in nearly every line, hinting that it's not a true `outflow' but just some excited region.... ].
Morphologically, it appears to be outflow-like, with bow-shaped ends, but the
feature appears over too narrow a range.  This feature could be the relic of an
older outflow that, now that it is no longer being actively inflated, is slowly
collapsing under external pressure.

[Alternatively, could the apparent bubble actually be a radiation-blown bubble, rather
than an outflow-driven bubble, that has slow-moving material along the edges?
This general sort of feature is seen in \citet{Rosen2016a}.]


\paragraph{The progression of star formation and its relation to Sgr B2 N}
Our observations, and those reported in \citet{Ginsburg2018a}, reveal several
compact fragments that are most likely YSOs extending along a streamer to the
northwest of the Sgr B2 N core.  There are dozens of candidate YSOs in that
direction, but only a few to the south of Sgr B2 N within $\sim0.5$ pc, though
there is gas and dust emission observed in that direction.  The lack of protostars
hints that star formation is progressing from north to south in this structure.

\paragraph{Pillars in Sgr B2 N}
There are several `pillar' structures like those seen in M16 spread throughout
the Sgr B2 N region.  These pillars are signposts pointing to nearby massive stars.


\section{SiO Maser in Sgr B2 M}
We have precisely located the SiO maser-emitting region in Sgr B2 M.  Previous
observations of this maser \citep{Morita1992a,Shika1997a} had poor astrometric
precision.
We detect the SiO v=1 J=2-1 line as a maser with peak intensity $\sim0.2$ Jy \perbeam.
We also detect masing in the $^{29}$SiO v=0 J=2-1 line.
No emission is detected from this source in the SiO v=2 J=2-1 line.


\section{Discussion}
The structures we see in Sgr B2 N are very similar to those shown in \citet{Rosen2016a}
for the turbulent simulation with radiative transfer included.

\end{document}
