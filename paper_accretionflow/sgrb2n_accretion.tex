Title: An extraordinary accretion flow in Sgr B2 North

\section{Introduction}

Sgr B2 N

High-mass protoclusters

High-mass stars


\section{Observations}


\section{Results}

There are several notable features that make this accretion flow unique within our Galaxy:

\begin{itemize}
    \item Four spiral-shaped accretion filaments point toward a common central hub
    \item The accretion filaments are \emph{optically thick} in the 1 mm continuum
        along their centers
    \item There is a central hypercompact \hii region that cannot be seen at 1 mm
        or shorter wavelengths, yet some of its light leaks out in H41$\alpha$ emission.
\end{itemize}


title?: The radiation environment in the most massive forming cluster in the local group

Toward Sgr B2 N, we observe a transition from likely optically thin to
certainly optically thick emission between 3 and 1 mm.  The overall structure
of the region, as seen at 3 mm, is a central main north-south filament, another
long and bright filament toward the east, and several smaller filaments at
various angles along the main ridge.  At 1 mm, there is a narrow region along
the main ridge with $T_{B,1 mm} < T_{B,3 mm}$, which is a clear indication
that the material is optically thick and centrally heated.

The central source, Sgr B2 N K2, is behind this veil.  At 3 mm, it has a peak
brightness temperature of $\sim900$ K toward the central source, but at this
location, the brightest position measures only $\sim200$ K.  (CHECK) The 1 mm 
source centroid is also offset from the 3 mm centroid, indicating that we are
only seeing a component or reflection of the illuminating source at 1 mm.
K2 is behind the veil.

K2 is also at the intersection between the main north-south filament and the
eastern filament.  It is reasonable to suppose that it is the central
accretion point for the whole $>10^4$ \msun system.

We are able to see some component of the K2 source at 1 mm, which indicates
either that the source is \emph{much} more luminous at that position (unlikely)
or that there is a small gap in the foreground cloud at that position.
In either case, though, we can see an absorption spectrum through the foreground
veil toward this line-of-sight.



\paragraph{Ruling out synchrotron emission} The spectral index from 3 mm to 1
mm is negative, with $\alpha\approx-1.6$.  This is steeper than any synchrotron
source (right?).
